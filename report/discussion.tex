\chapter{Discussion}

\section{Strengths and Limitations}
Our work rests on a solid foundation: we designed the methodology with care, used clear evaluation steps, and tracked progress with established metrics. We trained and tested the system on a large, varied dataset and put checks in place to keep bias at bay. In trials, our approach consistently hit—or even surpassed—its targets, which speaks to its practical value. \\ \vspace{5mm}

Still, no study is flawless. Our tests took place in controlled environments that can’t capture every real-world twist, and some simplifying design choices may narrow how widely our findings apply. We’ve also yet to tweak the system for real-time use, so time-sensitive applications could challenge performance. By calling out these points, we aim to frame our results honestly and point to areas where the next phase of work can step in.

\section{Possible Improvements}
\begin{itemize}
	
	\item \textbf{Try advanced algorithms:} Explore and compare additional heuristics (e.g., Manhattan, Euclidean, Octile) to improve search efficiency and reduce path cost. Use a more advanced preprocessing technique like contraction hierarchies, which is more often used in real world routers.
	
	\item \textbf{Optimize performance:} Speed up processing with parallel computing, model pruning, or GPU acceleration so the system can run faster and handle larger workloads.
	
	\item \textbf{Broaden evaluation:} Test the algorithm on new benchmarks and in real-world scenarios, such as user trials or field tests, to confirm its effectiveness under different conditions.
	
	\item \textbf{Improve usability:} Develop a simple interface or API and work on integration with existing platforms to make the solution easier to adopt and test by end-users.
	
	\item \textbf{Visualization improvements:} Enhance animation controls and graphical output to aid debugging and make the tool more intuitive for users.
	
	\item \textbf{Dynamic obstacle handling:} Integrate real-time map updates and re-planning so the algorithm adapts smoothly to changes in the environment.
	
\end{itemize}
