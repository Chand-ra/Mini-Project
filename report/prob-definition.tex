\chapter{Problem Definition}


%\section{Graph Representation}
%\begin{itemize}
%	\item Define the graph G=(V, E) where V is the set of vertices (nodes) and E is the set of edges.
%	\item Edge weights represent costs (e.g., travel time, distance, tolls).
%\end{itemize}

%\section{Objectives}

\section{Graph Representation}

\begin{enumerate}
	\item Road network can be represented as a graph $G(V, E)$, where $V$ is the set of vertices (nodes) and $E$ is the set of edges.
	\item Each edge is assigned a weight representing cost metrics such as travel time, distance, or implement traditional algorithms such as Dijkstra’s for single-source shortest path calculations and Bellman-Ford for graphs with negative edge weights tolls. 
	\item Store the graph efficiently using adjacency matrices.
\end{enumerate}

\section{Data Preprocessing}
\begin{enumerate}
	
	\item Load and clean large-scale datasets from OpenStreetMap.
	\item  Convert raw data into a structured graph representation, ensuring consistency in node labeling, edge definitions, and weight normalization. 
	\item Preprocess data using graph simplification techniques such as contraction hierarchies to speed up queries.
	
\end{enumerate}

\section{Choosing and Implementing an Algorithm}
\begin{enumerate}
	
	\item Implement Dijkstra’s for single-source shortest paths and Bellman-Ford for graphs with negative weights.
	\item Enhance performance using \( \mathbf{A^*} \) with heuristics for more efficient search convergence.
	\item Optimize graph traversal with bidirectional search to reduce search space.
	
\end{enumerate}

\section{Visualization and Analysis}

\begin{enumerate}
	\item Use visualization tools to represent shortest path calculations.
	\item Create comparative performance graphs showing execution time with different algorithms. 
\end{enumerate}
